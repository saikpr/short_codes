\chapter{Issues and Challenges}
Issues addressed in the program are:
\begin{itemize}
\item Even a slight change in the data will change the hash value of the data. Sometimes the file managing system of operating system adds data to files like thumbs, so those thumbs were removed by the code before checking hash value.
\item Rather than deleting duplicate files in just one directory the code also deletes duplicates in the sub directories.
\item For more interactive use a graphical user interface was designed for the user to input the directory path supposedly containing the duplicate files.
\item For fast checking the files were scanned in three phases comparing size in first instance, then hash value of first 1024 bytes and then hash of complete files so that unnecessary comparisons are avoided.
\end{itemize}
Challenges to be addressed by the program are:
\begin{itemize}
\item A better GUI can be included with added functionality in which one can scan files before deleting.
\item Certain corrupt duplicate files can also be compared by checking hash values of parts of file data.
\item Option can be provided for using either MD5 or SHA algorithms in case of extra security requirements.
\item Certain bugs in the listing algorithm which deletes same sized duplicates need to addressed.
\end{itemize}