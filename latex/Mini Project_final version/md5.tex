\cleardoublepage
\phantomsection
\addcontentsline{toc}{chapter}{References}
\chapter{The MD5 Algorithm}

\begin{itemize}
\item MD5 is generally used as a digital signature for various cases like online transaction etc.
\item MD5 takes an arbitrary length of message including files and produces a fingerprint called message digest of length 128 bit from the message.
\item It is conjectured to be computationally infeasible to produce messages having same message digest.
\item Utilised in cases where a secure compression of large files is required before encryption with a private key under public-key cryptographic system like PGP.
\end{itemize}
Suppose a b-bit message as input, and
Let the input be a b-bit message of which out message digest in to be printed.

\begin{enumerate}
\item Append padded bits:
\begin{itemize}
\item The message is padded so that its length is 448 mod 512.
\item  A single 1 bit is added to message and then 0 bits are appended so that the length in bits is 448 mod 512
\end{itemize}
\item Append length:
\begin{itemize}
\item  A 64 bit representation of b is appended to the result of the previous step.
\item  The resulting message has length that is multiple of 512 bits.
\end{itemize}
\item Initialize the message digest buffer:
\begin{itemize}
\item  A four word buffer is used to compute message digest. (A,B,C,D) each of 32 bit
\item These registers are initialized to values:
\begin{itemize}
\item  A: 01 23 45 67
\item  B: 89 ab cd ef
\item  C: fe dc ba 98
\item  D: 76 54 32 10
\end{itemize}
\end{itemize}
\item Process message:
\begin{itemize}
\item  F(X,Y,Z) = XY v NOT(X)Z
\item  G(X,Y,Z)= XZ v Y NOT(Z)
\item  H(X,Y,Z)= X xor Y xor Z
\item  I(X,Y,Z)= Y xor (X v NOT(Z))
\end{itemize}
\item  Process message in 16 word blocks.
\\ If bits of X,Y and Z are independent unbiased, then each bit of F,G,H,I will be independent and unbiased.
\item Step 6:
\begin{itemize}
\item  The message digest produced as output is A,B,C,D.
\item  Output begins with a lower order byte of A and end with Least significant byte of D.
\end{itemize}
\end{enumerate}