\chapter{BackGround}


The program makes use of hashing to compare two files. There are many file hashing mechanisms available.
\section{File hashing}: 
File hashing refers to the technique of mapping a particular variable length data to a fixed size value or data. These fixed data are referred as hash value , checksums etc. It is a very useful for handling large files whose data cannot be efficiently managed  for applications like file comparison and file searching.
Hash functions are also used to build caches for large files which are stored in slow media. Caches are easier to manipulate than hash values as they can be easily re-written for cases of collision(i.e. same cache value for different files). 
\section{Various file hashing techniques and algorithms are available}
\subsection{MD5}The MD5 message-digest algorithm is a widely used hash function producing a 128bit (16-byte) hash value, typically expressed in text format as a 32 digit hexadecimal number. MD5 has been utilized in a wide variety of cryptographic applications, and is also commonly used to verify data integrity.

\subsection{SHA-1}   SHA(Secure hash Algorithm) is a cryptographic hash function which produces 160 bit (20 bytes) hash value. The value is represented in form of 40 digit long hexadecimal number.

\subsection{SHA-2}   It consists of 6 hash functions with digests that are 224, 256, 384 or 512 bits. No collisions have been found yet for this hash function while the other two suffer from real and theoretical collisions however rare.


