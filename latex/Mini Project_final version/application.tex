\chapter{Application}
The hashing algorithm can be used in various cases.
\\~\\

\begin{itemize}
\item One can compare integrity of files using this algorithm.\\ 
          File duplicity can be removed comparing hash values.
\item  Certain related content about the files can be easily and efficiently searched online by simply passing the hash value as search argument.
\item  Hashing has also been used in cryptography in order to prevent unnecessary access and deciphering of sensitive data.
\item  Security can be increased efficiently by comparing a large data’s hash key to the another data hash key and using the data as passwords. If the hash values of data do not match then no need of comparing the data.
\item  Methods have been developed to embed large security data in digital keys and these keys can be hashed for searching appropriate match in database before actual comparison of keys.
\item	File hashing has been used in various other fields also to compare files with slight modification of data such as image processing, plagiarism check, video forgery and similar cases of duplicity.
\end{itemize}