\cleardoublepage
\phantomsection
\chapter{Attributes of clusters}

Computer clusters may be configured for different purposes ranging from general purpose business needs such as web-service support, to computation-intensive scientific calculations. In either case, the cluster may use a high-availability approach. Note that the attributes described below are not exclusive and a "compute cluster" may also use a high-availability approach, etc.
A load balancing cluster with two servers and N user stations

"Load-balancing" clusters are configurations in which cluster-nodes share computational workload to provide better overall performance. For example, a web server cluster may assign different queries to different nodes, so the overall response time will be optimized.[10] However, approaches to load-balancing may significantly differ among applications, e.g. a high-performance cluster used for scientific computations would balance load with different algorithms from a web-server cluster which may just use a simple round-robin method by assigning each new request to a different node.[10]

"Computer clusters" are used for computation-intensive purposes, rather than handling IO-oriented operations such as web service or databases.[11] For instance, a computer cluster might support computational simulations of vehicle crashes or weather. Very tightly coupled computer clusters are designed for work that may approach "supercomputing".

"High-availability clusters" (also known as failover clusters, or HA clusters) improve the availability of the cluster approach. They operate by having redundant nodes, which are then used to provide service when system components fail. HA cluster implementations attempt to use redundancy of cluster components to eliminate single points of failure. There are commercial implementations of High-Availability clusters for many operating systems. The Linux-HA project is one commonly used free software HA package for the Linux operating system.