\cleardoublepage
\phantomsection
\chapter{Sun Grid Engine}
Sun Grid Engine (SGE) is an advanced job scheduler which schedules jobs to run in a cluster environment. The main purpose of a job scheduler is to utilize system resources in the most efficient way possible.

SGE is written and distributed by Sun Microsystems under the Sun Industry Standards Source License, and is available free. \\
The Sun Grid Engine queuing system is useful when you have a lot of tasks to execute and want to distribute the tasks over a cluster of machines. For example, you might need to run hundreds of simulations/experiments with varying parameters or need to convert 300 videos from one format to another. Using a queuing system in these situations has the following advantages:

    Scheduling - allows you to schedule a virtually unlimited amount of work to be performed when resources become available. This means you can simply submit as many tasks (or jobs) as you like and let the queuing system handle executing them all.
    Load Balancing - automatically distributes tasks across the cluster such that any one node doesn’t get overloaded compared to the rest.
    Monitoring/Accounting - ability to monitor all submitted jobs and query which cluster nodes they’re running on, whether they’re finished, encountered an error, etc. Also allows querying job history to see which tasks were executed on a given date, by a given user, etc.


Grid Engine is typically used on a computer farm or high-performance computing (HPC) cluster and is responsible for accepting, scheduling, dispatching, and managing the remote and distributed execution of large numbers of standalone, parallel or interactive user jobs. It also manages and schedules the allocation of distributed resources such as processors, memory, disk space, and software licenses.
A typical Grid Engine cluster consists of a master host and one or more execution hosts. Multiple shadow masters can also be configured as hot spares, which take over the role of the master when the original master host crashes
%\includegraphics[width=0.75\columnwidth]{./cluster_diagram.gif} % Example image

    
   
    
