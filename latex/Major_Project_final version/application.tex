\chapter{Using SGE}


Running Interactive Jobs with SGE:

An interactive job is when you are running a program interactively on a node. This is good in the case of building/testing scripts, etc. This is not the place to run long running, very computationally intensive, or other jobs better suited to run in a batch job. An example would be the development of a matlab script. You can launch an interactive job, develop the script and write the job file. But when it comes to running the job itself, it needs to be submitted as a batch job. To run an interactive job, simply type qlogin
\\~\\
Running Parallel Jobs with SGE:\\
 A parallel job is where a single job is run on many nodes in an interconnected fashion, generally using MPI to communicate in between individual processes. If you are running the same program on the cluster as you would on your desktop, chances are you will want to use a serial job, not a parallel job. Parallel jobs generally are only for specially designed programs which will only work on machines with cluster management software installed.

Also not just any program can run in parallel, it must be programmed as such and compiled against a particular mpi library. In this case we build a simply program that passes a message between processes and compile it against the OpenMPI, the main mpi library of the cluster.