\cleardoublepage
\phantomsection
\addcontentsline{toc}{chapter}{References}
\chapter{Introduction}

 The exact definition of a cluster computer will depend a little on who you ask. However, there are some general characteristics that most will agree upon.
\begin{itemize}
\item Consists of many of the same or similar type of machines
\item (Heterogenous clusters are a subtype, still mostly experimental)
\item Tightly-coupled using dedicated network connections
\item All machines share resources such as a common home directory
\item All machines share resources such as a common home directory
\item (NFS can be a problem in very large clusters, so binaries and data must be pushed to scratch on each node.)
\item  They must trust each other so that rsh or ssh does not require a password,
    otherwise you would need to do a manual start on each machine.
\item Must have software such as an MPI implementation installed to allow programs to be run across all nodes 
\end{itemize}
In general, distributed computing is any computing that involves multiple computers remote from each other that each have a role in a computation problem or information processing.
\begin{itemize}
\item In business enterprises, distributed computing generally has meant putting various steps in business processes at the most efficient places in a network of computers. In the typical transaction using the 3-tier model, user interface processing is done in the PC at the user's location, business processing is done in a remote computer, and database access and processing is done in another computer that provides centralized access for many business processes. Typically, this kind of distributed computing uses the client/server communications model.
\item More recently, distributed computing is used to refer to any large collaboration in which many individual personal computer owners allow some of their computer's processing time to be put at the service of a large problem. The best-known example is the SETI@home project in which individual computer owners can volunteer some of their multitasking processing cycles (while concurrently still using their computer) to the Search for Extraterrestrial Intelligence ( SETI ) project. This computing-intensive problem uses your computer (and thousands of others) to download and search radio telescope data.
\end{itemize}
One of the first uses of distributed computing was the breaking of a cryptographic code by a group that is now known as distributed.net.
%\includegraphics[width=0.75\columnwidth]{./cluster_diagram.gif} % Example image

    
   
    
