\chapter{Code}
\begin{lstlisting}

import os
import sys
import stat
import md5
import wx

filesBySize={}      
# creating a dictionary to store files by size

def runp(event):
    global filesBySize
    filesBySize={} 

    def walker(arg, dirname, fnames):     
    # Funtion to traverse the directories
        
        d = os.getcwd()
        print d
        print dirname
        os.chdir(dirname)
        try:
            fnames.remove('Thumbs')
        except ValueError:
            pass        
        for fil in fnames:
            if not os.path.isfile(fil):
                continue
            size = os.stat(fil)[stat.ST_SIZE]
            if size < 100:
                continue
           
            if arg.has_key(size):
                a = arg[size]
            else:
                a = []
            
                arg[size] =a
            a.append(os.path.join(dirname, fil))
        os.chdir(d)

      
    x=content1.GetValue()
    print 'Scanning directory for same file size  "%s"....' % x
    os.path.walk(x, walker, filesBySize)    

    print 'Finding potential dupes...'
    potentialDupes = []
    potentialCount = 0
    trutyp = type(True)
    sizes = filesBySize.keys()
    sizes.sort()                                         
    #Sorting by file size
    for k in sizes:
        infil = filesBySize[k]
        outFiles = []
        hashes = {}
        if len(infil) is 1: continue
        print 'Testing %d files of size %d...' % (len(infil), k)
        for filnam in infil:                           
        # Hashing the first 2048 bytes of data
            if not os.path.isfile(filnam):
                continue
            dfile = file(filnam, 'rb')
            hasher = md5.new(dfile.read(2048))
            hashValue = hasher.digest()
            if hashes.has_key(hashValue):
                x = hashes[hashValue]
                if type(x) is not trutyp:
                    outFiles.append(hashes[hashValue])
                    hashes[hashValue] = True
                outFiles.append(filnam)
            else:
                hashes[hashValue] = filnam
            dfile.close()
        if len(outFiles):
            potentialDupes.append(outFiles)
            potentialCount = potentialCount + len(outFiles)
    del filesBySize

    print 'Found %d sets of potential duplicates...' % potentialCount
    print 'Scanning for real duplicates...'

    dupes = []
    for aSet in potentialDupes:
        outFiles = []
        hashes = {}
        for filnam in aSet:                           
        # Checking Whole file by hashing 4096 bytes each time from 
            print 'Scanning file "%s"...' % filnam    
            # list of suspected duplicate files
            dfile = file(filnam, 'rb')
            hasher = md5.new()
            while True:
                r = dfile.read(4096)
                if not len(r):
                    break
                hasher.update(r)
            dfile.close()
            hashValue = hasher.digest()
            if hashes.has_key(hashValue):
                if not len(outFiles):
                    outFiles.append(hashes[hashValue])
                outFiles.append(filnam)
            else:
                hashes[hashValue] = filnam
        if len(outFiles):
            dupes.append(outFiles)

    i = 0
    for d in dupes:
        print 'Original is %s' % d[0]
        for fi in d[1:]:
            i = i + 1
            print 'Deleting %s' % fi
            os.remove(fi)
        print
app=wx.App()                                    
# Creating GUI for the program using wxPython
win=wx.Frame(None,title="File Remover",size=(300,80))
bkg=wx.Panel(win)
remv=wx.Button(bkg, label="remove")
remv.SetBackgroundColour('blue')
remv.Bind(wx.EVT_BUTTON, runp)

content1= wx.TextCtrl(bkg)
hbox=wx.BoxSizer()
hbox.Add(content1,proportion=1,flag=wx.EXPAND ,border=5)
hbox.Add(remv,proportion=0,flag=wx.EXPAND ,border=5)
bkg.SetSizer(hbox)
win.Show()
app.MainLoop()

\end{lstlisting}
\vspace{3.5in}
 \textbf {Note :}
This Code has been Writtten in Python 2.7.6.
In this the Bug related with the file-opening parameter was fixed by using `rb' mode instead of `r' mode.

Also ,
It uses os.path.walk, which is to be deprecated in new versions nad replaced by os.walk.
